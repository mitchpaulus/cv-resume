%%%%%%%%%%%%%%%%%%%%%%%%%%%%%%%%%%%%%%%%%
% Medium Length Graduate Curriculum Vitae
% LaTeX Template
% Version 1.1 (9/12/12)
%
% This template has been downloaded from:
% http://www.LaTeXTemplates.com
%
% Original author:
% Rensselaer Polytechnic Institute (http://www.rpi.edu/dept/arc/training/latex/resumes/)
%
% Important note:
% This template requires the res.cls file to be in the same directory as the
% .tex file. The res.cls file provides the resume style used for structuring the
% document.
%
%%%%%%%%%%%%%%%%%%%%%%%%%%%%%%%%%%%%%%%%%

%----------------------------------------------------------------------------------------
%	PACKAGES AND OTHER DOCUMENT CONFIGURATIONS
%----------------------------------------------------------------------------------------

\documentclass[margin]{res} % Use the res.cls style, the font size can be changed to 11pt or 12pt here

\usepackage{helvet} % Default font is the helvetica postscript font
\usepackage{setspace}
\usepackage{bibentry}
%%\usepackage{geometry}
\nobibliography*
%\usepackage{newcent} % To change the default font to the new century schoolbook postscript font uncomment this line and comment the one above
%\usepackage{palatino}
%\usepackage[default,light,bold]{sourceserifpro}
%\usepackage[urw-garamond]{mathdesign}
\usepackage{lmodern}
\usepackage[T1]{fontenc}
%% \setlength{\resumewidth}{5.2in}
%% \setlength{\textwidth}{5.2in} % Text width of the document
\setlength{\parskip}{7pt}

\begin{document}

%----------------------------------------------------------------------------------------
%	NAME AND ADDRESS SECTION
%----------------------------------------------------------------------------------------

\moveleft.5\hoffset\centerline{\large\bf Mitchell T. Paulus} % Your name at the top

\moveleft\hoffset\vbox{\hrule width\resumewidth height 1pt}\smallskip % Horizontal line after name; adjust line thickness by changing the '1pt'

\moveleft.5\hoffset\centerline{1013 Mapleleaf Ln.} % Your address
\moveleft.5\hoffset\centerline{Coppell, TX 75019}
\moveleft.5\hoffset\centerline{(262) 483--9068, paulusm14@gmail.com}

%----------------------------------------------------------------------------------------

\begin{resume}

%----------------------------------------------------------------------------------------
%	OBJECTIVE SECTION
%----------------------------------------------------------------------------------------

% \section{OBJECTIVE}

% Motivated college student with special interests in Mechanical Engineering and experience with food service, retail, and personal services. Proven track record of excellent performance and advancement.

%----------------------------------------------------------------------------------------
%	EDUCATION SECTION
%----------------------------------------------------------------------------------------

\section{EDUCATION}

\underline{{\textit{Ph.D. Mechanical Engineering,} Texas A\&M University, College Station}}\hfill 2012--2017

\parbox{11cm}{Dissertation title: ``Active Remote Setpoint Optimization Utilizing BAS Trend Data''}

\underline{{\sl M.S. Engineering,} Milwaukee School of Engineering} \hfill 2010--2012

\parbox{11cm}{Thesis title: ``Measurement, Modeling, Analysis and Reporting Protocols for Short-Term Monitoring and Verification of Whole Building Energy Performance''}

\underline{{\sl B.S. Mechanical Engineering,} Milwaukee School of Engineering} \hfill 2006--2010

\section{PROFESSIONAL CERTIFICATION}

Professional Engineer (Mechanical) \#135214, TX \hfill 2019


\vspace{0.1in}

\section{EXPERIENCE}

{\textit{Energy Engineer} \hfill Sep 2017--Present} \\
Command Commissioning, LLC., Irving, TX

\begin{itemize}\itemsep -2pt % Reduce space between items
\item Conduct remote assessments and develop energy efficiency measures for Continuous Commissioning\textsuperscript{\copyright{}}.
\item Fault detection and diagnostics for existing building commissioning.
\item Develop enhanced sequences of operation along with measurement and verification of energy efficiency savings.
    \item Lead development of data-driven analytics in existing building
        commissioning and Continuous
        Commissioning\textsuperscript{\textregistered{}}.
\end{itemize}


{\sl Postdoctoral Research Associate}  \hfill Sep 2017--Present \\
{\sl Graduate Assistant Researcher} \hfill Sep 2012--Sep 2017 \\
Energy Systems Laboratory, College Station, TX

\begin{itemize} \itemsep -2pt % Reduce space between items
    \item Lead developer for CC-Compass (cc-compass.tamu.edu), a building automation system (BAS) data
        analytics web application. The web application is developed
        under the ASP.NET MVC 5 architecture, with C\# as the primary
        computational language. Manage data for 275,000+ sensors and
        over 11 billion data points.
    \item Developed algorithms for automated monitoring and verification of energy conservation projects and characterization of building component control sequences.
    \item Conducted research concerning fault detection and diagnostic algorithms for commercial buildings, along with monitoring and verification procedures.
\end{itemize}

\vspace{10pt}

{\sl Research Assistant} \hfill Sep 2009--Feb 2012  \\
Milwaukee School of Engineering, Milwaukee, WI
\begin{itemize}\itemsep -2pt
\item Contracted through ASHRAE to complete research on the subject of short-term monitoring for long-term building energy prediction (RP-1404).
\item Completed model development and presented findings at semi-annual ASHRAE conferences.
\end{itemize}


\section{PUBLICATIONS}
{\sl \textbf{Peer-Reviewed Journal Articles}}

\bibentry{Paulus2017}

\bibentry{Paulus2016c}

\bibentry{Paulus2016b}

\bibentry{Paulus2016a}

\bibentry{Paulus2015AlgorithmModel}

\bibentry{Gangisetti2015}

%\bibentry{Paulus2012}


{\sl \textbf{Book Chapters}}

Jan L.M. Hensen and Roberto Lamberts. \textit{Building Performance Simulation
for Design and Operation, Expanded Second Edition}. David E. Claridge
and \textbf{Mitchell T. Paulus}. Chapter 13: \textit{Building Simulation
for Practical Operational Optimization}.

David Strong and Victoria Burrows. \textit{A Whole-System Approach to High-Performance Green Buildings}. David E. Claridge, \textbf{Mitchell T. Paulus}, Charles Culp, and Kevin Christman, Chapter: \textit{Continuous Commissioning\textsuperscript{\textregistered{}}}, 2016

\textbf{\textit{Conference Presentations}}

``An Hourly Hybrid Multivariate Change Point Inverse  Model Using Short-Term Data for Annual Prediction of Building Energy and Performance: Results and Analysis'',\newline
2017 ASHRAE Winter Conference, Las Vegas, Seminar 62, Jan 2017

\textbf{\textit{Invited Presentations}}

``Software Skills/Tools You Should Know About as an Engineer'', ASHRAE Student Branch Meeting, Mar 21, 2018

``Software Designed for the Continuous Commissioning Process'', ASHRAE Student Branch Meeting, Sep 15, 2015

``Software to Assist in the CC Process'', Houston ASHRAE Chapter Student Night, Mar 2015
%----------------------------------------------------------------------------------------

\section{TEACHING EXPERIENCE}
{\sl Graduate Teaching Fellow }\newline
Instructor of Record for MEEN 315, Principles of Thermodynamics \hfill Spring 2017

{\sl Teaching Assistant}\newline
Energy Management in Commercial Buildings, MEEN 664, Texas A\&M \hfill \parbox[t]{1.5cm}{\raggedleft Fall 2016 \\ Fall 2015} \par

\vspace{10pt}

Application of Energy Management, MEEN 665, Texas A\&M \hfill \parbox[t]{2cm}{\raggedleft Spring 2017 \\ Spring 2016}

\vspace{10pt}
{\sl Guest Lectures}
\begin{itemize}
\item ``Time Constants and Measurements in Buildings'' \hfill Feb 21, 2017
\item ``Ventilation and Economizers in Buildings'' \hfill Sep 20, 2016
\item ``Buildings as RC Model and Thermal Zoning'' \hfill Sep 15, 2016
\item ``Level 0 Continuous Commissioning Measures'' \hfill Mar 1, 2016
\item  \parbox{10cm}{``Software Tools in the Continuous Commissioning \\ Process''} \hfill Feb 4, 2016
\item ``Calibrating Building Simulations'' \hfill Oct 2015
\item ``Runge-Kutta Methods for Numerical Heat Transfer'' \hfill Jan 2012
\end{itemize}

\medskip


{\sl Tutoring} \newline
Milwaukee School of Engineering \hfill 2011--2012

\section{LEADERSHIP EXPERIENCE}

President and Webmaster, Texas A\&M ASHRAE Student Branch \hfill 2015--2016

Member, Texas A\&M Mechanical Engineering Leadership Council \hfill 2014--2016

Head Coach, Texas A\&M Men's Club Volleyball \hfill 2014--2015\

President, MSOE ASHRAE Student Branch \hfill 2010--2011

\section{AWARDS}
Texas A\&M Graduate Teaching Fellow, 2017

Fall MSOE Graduation Class Respondent (Valedictorian), 2010

ASHRAE Houston Scholarship, 2014, 2015, 2016

WFIC Ladish Co. Scholarship, 2009

Regal Scholarship, 2006

\section{SOFTWARE}

Computer Programming: C\#, shell, Matlab, Mathematica, VBA for Microsoft Office, Python

Building Simulation: \parbox[t]{16cm}{ EnergyPlus, eQuest, Carrier's
HAP, WinAM, Inverse Modeling \\ Toolkit }

Web Development: \parbox[t]{20cm}{Model-View-Controller Architecture,
html/css/javascript, jQuery,\par knockout.js, d3.js}

Database Management: SQL Server Management Studio

Solid Modeling: ANSYS, Pro Engineer, SolidWorks/SolidEdge

Other: Version control in git and bazaar, \LaTeX, GIMP (image processing), Vim

\section{PROFESSIONAL SOCIETIES}

\parbox[t]{9cm}{American Society of Heating, Refrigerating, and Air-Conditioning Engineers (ASHRAE)} \hfill \parbox[t]{3cm}{\raggedleft Stu. Member \\ 2011--Pres.}

\parbox[t]{9cm}{ASHRAE Technical Committee 4.7, Energy Calculations }\hfill \parbox[t]{3cm}{\raggedleft Corres. Member \\ 2015--Pres.}

\section{OUTREACH}
\parbox[t]{10cm}{Contributed to 2017 ASHRAE Fundamentals Handbook, \textit{Energy Estimating and Modeling Methods} Chapter}

Texas A\&M Freshman Student Informational \hfill Dec 2015\\
\strut \hfill Mar 2016

Texas A\&M Mechanical Engineering Leadership Council \hfill 2015--2017


\section{RESEARCH INTERESTS}

Energy efficiency and conservation, particularly focused on buildings

Web application development

Continuous Commissioning\textsuperscript{\textregistered{}}

Data-driven building energy modeling

\section{REVIEWING}

Energy and Buildings

%% Science and Technology for the Built Environment

Journal of Building Performance Simulation

Building Simulation

ASHRAE Conference Papers

\end{resume}
\bibliographystyle{plain}
\nobibliography{Thesis}
\end{document}
